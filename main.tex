\documentclass[12pt]{article}

% Basic packages
\usepackage[utf8]{inputenc} % For UTF-8 encoding
\usepackage[T1]{fontenc}    % For better font encoding
\usepackage{lmodern}        % For modern fonts
\usepackage{amsmath}        % For advanced math
\usepackage{amssymb}        % For additional math symbols
\usepackage{graphicx}       % For including images
\usepackage{geometry}       % For page layout
\usepackage{hyperref}       % For hyperlinks
\usepackage{caption}        % For customizing captions
\usepackage{float}          % For better control of figure placement


\newcommand{\PD}[1]{\color{red}#1}      % Command for Puspita
\newcommand{\CP}[1]{\color{blue}#1}     % Comment command for Chloe
\newcommand{\AG}[1]{\color{green}#1}    % Comment command for Andrea


% Document starts here
\begin{document}

\title{Your Title Here}
\author{Your Name}

\maketitle
\begin{abstract}

\end{abstract}



\section{Introduction}



\section{Design and Characterization}
In this section we present the different units that can be used to assemble the continuum robot. 
Each unit is characterized in terms of allowed motion (DoFs) and the corresponding stiffness.

\subsection{Unit 1: Name of the unit}
\subsection{Unit 2: Name of the unit}
\subsection{Unit 3: Name of the unit}
\subsection{Unit 4: Name of the unit}
\subsection{Unit 5: Name of the unit}
\subsection{Unit 6: Name of the unit}



\section{Fabrication}
In this section we present the fabrication procedure, the required material and machines and an estimate of the cost of the different units.

\subsection{Fabrication Procedure}
Here we describe how the units are assembled and the different steps to fabricate them.

\subsection{Material and Machines}
Here we describe the different materials and machines used to fabricate the units.
We should focus on the parameters of the 3D printers so that if someone has a different printer, they can adapt the parameters accordingly.

\subsection{Cost}
Here we provide an estimate of the cost of the different units.

\section{Applications}
In this section we present different application showcasing the versatility of the different units.

\subsection{Pick object from inside a computer case}
This is a simple application the robot is used to pick up an object (screwdriver) from inside a computer case.
The interest is in the fact that we could allow contact with the computer case and components inside the case as the robot is compliant.
One cool thing could be to use the same computer that controls the robot (careful if we use a screwdriver which is conductive).

\subsection{Do something inside the laser cutter}
In this application we could showcase a tedius tasks to perfom in a more industrial environment.
We might take the opportunity to show the limitation of some of the units and how replacing some of them changes the robot capabilities.




\section{Conclusion}
Write your conclusion here.


\end{document}